% Created 2016-12-15 Thu 11:32
\documentclass[10pt,conference,compsocconf]{IEEEtran}
\usepackage[utf8]{inputenc}
\usepackage[T1]{fontenc}
\usepackage{fixltx2e}
\usepackage{graphicx}
\usepackage{grffile}
\usepackage{longtable}
\usepackage{wrapfig}
\usepackage{rotating}
\usepackage[normalem]{ulem}
\usepackage{amsmath}
\usepackage{textcomp}
\usepackage{amssymb}
\usepackage{capt-of}
\usepackage{hyperref}
\usepackage{bm}
\usepackage{svg}
\usepackage{graphicx}
\graphicspath{{pics/}}
\usepackage[margin=1in]{geometry}
\usepackage{algorithm}
\usepackage{algpseudocode}
\documentclass[10pt,conference,compsocconf]{IEEEtran}
\author{Laurent Lejeune, Tatiana Fountoukidou, Guillaume de Montauzon}
\date{\today}
\title{Group 97: Road Segmentation}
\hypersetup{
 pdfauthor={Laurent Lejeune, Tatiana Fountoukidou, Guillaume de Montauzon},
 pdftitle={Group 97: Road Segmentation},
 pdfkeywords={},
 pdfsubject={},
 pdfcreator={Emacs 25.1.1 (Org mode 8.3.6)}, 
 pdflang={English}}
\begin{document}

\maketitle
\section{Data exploration}
\label{sec:orgheadline1}
The provided training set contains 100 images of size 400x400 along with their ground-truth. A total of 6 images are discarded because they either show a too small quantity of positive class pixels. 
Most images are made of grid-like roads, sometimes occluded by trees. 
\section{Mid-level segmentations}
\label{sec:orgheadline2}
The image pixels are first grouped in two different manners:
\begin{enumerate}
\item Square patches: The image is divided in non-overlapping patches of size 16x16.
\item SLIC Superpixels (Simple Linear Iterative Clustering) \cite{achanta12}: Pixels are grouped in mid-level regions in an iterative manner. The algorithm starts from a regular grid of cluster centers and iteratively updates the labels of their neighboring centers based on a distance measure. This method improves over the square patches method because the pixels are already pre-segmented. Their feature vector will therefore be easier to discriminate.
\end{enumerate}
\section{Feature extraction}
\label{sec:orgheadline3}
Following an exploration of the related litterature, we select a set of features that will be extracted.
\begin{itemize}
\item SIFT (Scale-Invariant Feature Transform) \cite{lowe99}: This descriptor is used extensively in computer-vision applications. It computes a histogram of oriented gradients on 16x16 windows centered at a keypoint and gives a descriptor of 128 scalar values. The keypoint detection step is not performed, instead we extract the descriptors on a dense grid and encode them in a "bag-of-features" manner through the following steps: 
\begin{itemize}
\item Based on a sufficiently large number of SIFT descriptors computed on 10 images, we start by fitting a PCA model. We have checked that the explained variance at 60 components is above \(99%\).
\item A codebook is generated on the aforementioned training samples. A codebook is merely a set of K-means clusters that is used to encode the input (compressed) descriptors to integer values.
\begin{itemize}
\item We then compute a normalized histogram of codes (bag-of-features) in each patch/superpixels. This gives us a single texture feature vector for mid-level regions.
\end{itemize}
\end{itemize}
\item Hough line transform: This transformation has already been used in a state-of-the-art method \cite{2016ISPAr41B3..891L}. First, the edge map is computed using a canny edge detector. Given some parameters, a set of lines are extracted on the edge maps and sorted based on their RGB variance, i.e. we want to keep the lines along which the color variations is minimal.
\item Euclidean distance transform. This straightforward transform is used to compute, at each pixel location, the shortest "taxicab" distance to an edge pixel. Again, a canny edge map is used as input.
\end{itemize}
\section{Methods}
\label{sec:orgheadline5}
Two (several) methods have been implemented and tested. A Conditional Random Field approach will provide a baseline. It will be compared to a Convolution Neural Network appraoch.
\subsection{Conditional Random Field using Structured Support Vector Machine}
\label{sec:orgheadline4}
Using a Conditional Random Field model, one can exploit the spatial relations between mid-level regions. Indeed, a patch considered as road gives a strong prior to the "roadness" of its neighboring patches. Our approach consists in predicting the class probabilities of segments

\section{Litterature review}
\label{sec:orgheadline11}
\subsection{Road Segmentation in Aerial Images by Exploiting Road Vector Data \cite{6602035}}
\label{sec:orgheadline6}
\subsection{Morphological road segmentation in urban areas from high resolution satellite images \cite{gaetano:inria-00618222}}
\label{sec:orgheadline7}
\subsection{Connected Component-Based Technique for Automatic Extraction of Road Centerline in High Resolution Satellite Images \cite{sujatha15_connec_compon_based_techn_autom}}
\label{sec:orgheadline8}
\subsection{Machine Learning Based Road Detection from High Resolution Imagery}
\label{sec:orgheadline9}
\subsection{Road Extraction Using K-Means Clustering and Morphological Operations \cite{maurya2011road}}
\label{sec:orgheadline10}

\bibliographystyle{ieeetr}
\bibliography{refs}
\printbibliography
\end{document}